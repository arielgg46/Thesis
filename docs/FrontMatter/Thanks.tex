\begin{acknowledgements}
Quiero agradecer, en primer y destacado lugar, a toda mi familia, por ser mi apoyo vital, emocional y formador. Solo lo que me han otorgado, con enorme amor y sacrificios, ha hecho posible que llegue tan lejos en la vida.

A mi madre, que fue mi primera tutora, que veló mis primeros pasos en la programación, adentrándome en este mundo tan hermoso, y que me ha acompañado de la mano todo el camino. Ponerla orgullosa de mí motiva mi búsqueda de la perfección, en todo lo que hago.

A mi padre, que me enseñó a ser fuerte, a mantenerme positivo en todo momento y, en las situaciones más difíciles, aguantar y seguir avanzando. De él aprendí que cualquier cosa que me propongo, lo hago bien o no lo hago.

A mi hermana, mi cotutora, mi modelo a seguir. Por siempre desempeñar con excelencia el papel de la hermana mayor, por ser mi bastón ético y moral, y por hacerme reir cuando lo necesitaba.

A mis abuelos, por cuidarme, sostenerme, y mimarme como solo ellos saben hacer.

A mis amigos cercanos, que he tenido la suerte de tener muchos y muy buenos. Si los listo, siempre habrá aquel que se quedó fuera y se pondrá celoso, así que ya me aseguraré de agradecerles en persona. Por estar ahí cuando los necesité, por ser mi fuente inacabable de experiencias, motivación y el equilibrio personal durante esta y otras etapas, y bueno, por las risas.

A mis tutores, por confiar en mi trabajo, por su guía experta, paciencia, atención y apoyo constante a lo largo de este proceso.

A mis profesores de todas las enseñanzas. Ellos construyeron la base académica y docente que me llevó a este punto. Un agradecimiento particular a Lupe, que motivó mi amor por la matemática, y la enseñanza.

Finalmente, a las instituciones y fuentes de financiamiento que hicieron posible el acceso a recursos y herramientas necesarios para llevar adelante esta investigación. A los autores de los trabajos científicos que la anteceden y sobre los que se basa.

Muchísimas gracias a todos.

\end{acknowledgements}